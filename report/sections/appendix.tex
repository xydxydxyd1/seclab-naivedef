\section{Appendix}%
\label{sec:Appendix}

Mostly notes to myself.

\subsection{Improvements}

The following improvements can be made to the experiment

\paragraph{Conflict analysis} A thorough analysis of how the model handles
conflicting instructions with respect to formatting, placement, and wording may
be insightful.

\paragraph{More defense prompts} More defense prompts with different formatting,
wording, and placement should be tested.

\paragraph{Sophisticated attacks} Ignore attacks are too simple. More
sophisticated attacks should be tested.


\subsection{Additional Notes}

\paragraph{Sources} This study is inspired by the paper on Instruction
Hierarhcy. A lot of stuff mentioned is from there.

\paragraph{Machine generated tasks} The two instruction selected may not be
comprehensive enough. However, while a machine generated task will ensure that
enough of a variety of tasks are tested to cover more use cases, since the
counting task imply that the model processes the text, it is good enough for me.
As the TensorTrust paper mentioned, the goal of most language models is to
process the inputs without shutting down the model. This will not be a focus.

\subsection{Archive}

\emph{This section contains misc. text snippets from previous renditions saved
in case they are needed later.}

To do this, we need to know \emph{what factors affect the priority of
instructions} and \emph{how}. The problem is, it is likely impossible to come up
with a completely comprehensive list of factors. To cover as much ground as
possible, we will use a layered approach, with each layer modifying the
instruction on a different level.

\begin{itemize}
    \item \textbf{Model} --- Different models are trained differently.
    \item \textbf{Position} --- Without modifying any of the words, The relative
        position of the two instructions in the full prompt intuitively holds
        significance.
    \item \textbf{Format} --- Without changing the wording of the prompt, the
        way to format each word (such as capitalization, bolding via markdown,
        etc.) may impact priority.
    \item \textbf{Wording} --- Without changing the structure of the prompt,
        the way to word each instruction, including the use of keywords like
        "important", punctuation, etc. may impact priority.
    \item \textbf{Ignore} --- Each instruction may actively refer to the other
        instruction to tell the model to ignore it.
\end{itemize}

Model and position are relatively simple. Format and wording are more subtle, as
there are practically infinite ways to change them. Ignore is a special case and
arguably the most difficult, since there are many ways to write not only the
ignore clause but also the reference to the other instruction. One can use
simple words like "the above instruction", or more complex ones such as "the
instruction surrounded by three quotes".

\subsection{Ignore}

\emph{Ignore} adds clauses to each instruction to tell the model to ignore the
other instruction. This differs from the other methods because it is the only
one that acknowledges the other instruction's existence.
