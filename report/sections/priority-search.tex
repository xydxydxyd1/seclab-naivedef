\section{Priority search}%
\label{sec:Priority search}

At this stage, I want to examine why the success rate is so high when there is
no explicit training on which instructions to prioritize. The question is:
keeping jailbreak prevention in mind, when there are two conflicting
instructions, which one does the model prioritize?


\subsection{Hypotheses}

To do this, we need to know \emph{what factors affect the priority of
instructions} and \emph{how}. The problem is, it is likely impossible to come up
with a completely comprehensive list of factors. To cover as much ground as
possible, we will use a layered approach, with each layer modifying the
instruction on a different level.

\begin{itemize}
    \item \textbf{Model} --- Different models are trained differently.
    \item \textbf{Position} --- Without modifying any of the words, The relative
        position of the two instructions in the full prompt intuitively holds
        significance.
    \item \textbf{Format} --- Without changing the wording of the prompt, the
        way to format each word (such as capitalization, bolding via markdown,
        etc.) may impact priority.
    \item \textbf{Wording} --- Without changing the structure of the prompt,
        the way to word each instruction, including the use of keywords like
        "important", punctuation, etc. may impact priority.
    \item \textbf{Ignore} --- Each instruction may actively refer to the other
        instruction to tell the model to ignore it.
\end{itemize}

Model and position are relatively simple. Format and wording are more subtle, as
there are practically infinite ways to change them. Ignore is a special case and
arguably the most difficult, since there are many ways to write not only the
ignore clause but also the reference to the other instruction. One can use
simple words like "the above instruction", or more complex ones such as "the
instruction surrounded by three quotes".

\subsection{Code}

First


\paragraph{Ignore} The problem is, the conflicting instruction actively tells to
ignore the other one. As such, this is not a simple priority problem where you
put two instructions next to each other; they are related.
